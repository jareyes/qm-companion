\setchapterpreamble[u]{\margintoc}
\chapter{Mathematical Background}

The character of quantum mechanics is to a large extent dependent on the mathematics used to describe it. At the end of the day, all theory needs to be supported by experimental evidence. That is, by actual measurements. We quantify our measurements with numbers to speak in a standardized, precise way. When we ask questions like, ``How long?'' or ``How much?'', we typically use numbers to give an answer.

In quantum mechanics, we will use increasingly sophisicated mathematical constructs to describe the world around us, including experiments physicists perform. Some of properties of numbers that we take for granted will remain, while others will disappear. Here we begin with a exploration of two particular useful number systems: the \emph{real} and the \emph{complex} numbers.

\section{Real Numbers}\marginnote{\textbf{TODO:} Real numbers. Communitivity, distributivity. (Figures showing geometric equivalence.) \textbf{Exercises.} Field properties. Example, the real numbers. Binary. Finite fields. $\mathbf{Z}_2$. $V_4$. \textbf{Exercises.} Cartesian plane. Absolute value. Distance. Circles. \textbf{Exercises.}}

We begin with an old friend, the \emph{real line}. When we talk about the real line, we're really just talking about an ordinary geometric line. To define a sense of order, we label the points on the line with the set $\mathbf{R}$ of all \emph{real numbers}. That is, we identify each point on the line with a unique real number. And  we often speak as if the numbers were actually on a line, and as if the points on the line are actually numbers.

\begin{figure}[h]
\begin{scaletikzpicturetowidth}{\textwidth}
\begin{tikzpicture}[scale=\tikzscale]
\draw[thick, <->] (-3.5,0) -- (3.5,0);
\foreach \x in  {-3,-2,-1,0,1,2,3,-0.333, 0.5, 1.4142, 2.7183, 3.1416}
\draw[shift={(\x,0)}, fill=black] (0,0) circle (1pt);
\foreach \x in {-3,-2,-1,0,1,2,3}
\draw[shift={(\x, 0)}] node[below=3pt] {\small $\x$};
\foreach \x/\xlabel in {-0.333/{-1/3}, 0.5/{1/2}, 1.4142/\sqrt{2}, 2.7183/e, 3.1416/\pi}
\draw[shift={(\x, 0)}] node[above=3pt] {\small $\xlabel$};
\end{tikzpicture}
\end{scaletikzpicturetowidth}

\caption{The real line}
\end{figure}

You may be wondering what makes the real numbers ``real''? The term real is unfortunate. The existence of real numbers is no more or less certain than any other kind of number---including the so-called imaginary numbers that we will meet later in this chapter. The philosopher Ren\'e Descartes came up with the term while investigating roots of polynomials in the 17th century. He was trying to distinguish different kinds of solutions. Since then, the name stuck. And it has confused students of mathematics ever since.

In some sense, the complex numbers, which include the combination of real and imaginary numbers, are much more real---or at least more complete---than the real numbers alone. The \emph{Fundamental Theorem of Algebra} guarantees that all polynomials of a single variable have a solution which is complex number. For example, the equation

\begin{equation}\label{eqn:mb-imaginary} x^2+ 1 = 0 \end{equation}

has no real number solution. We will see later that there is a complex number solution to that equation. Because complex solutions always exist for single-variable polynomials with complex coefficients, we say that the complex numbers are \emph{algebraically closed}.\sidenote{We will see a more examples of numbers that are closed under other operations soon.} The real numbers are not. The fact that equation \ref{eqn:mb-imaginary} has no real number solution is witness to the fact that the real numbers are not algebraically closed.

While the real line may look simple, do not be fooled. Its internal structure is incredibly complicated. For now we'll call out a few subsets of $\mathbf{R}$ that are so commonly used that they get their own symbols as well. They are the natural numbers $\mathbf{N}$, the integers $\mathbf{Z}$, and the rational numbers $\mathbf{Q}$.\sidenote{The \emph{Zahlen} means ``number'' in German. German mathematicians used $\mathbf{Z}$ to refer to integers in their papers. The rest is history.}\sidenote{Rational numbers are ratios of integers, AKA \emph{quotients}.}

\subsection{Natural Numbers}
The natural numbers $\textbf{N}$ are also sometimes called the counting numbers. They are the ones we first learn. They start 1, 2, 3, and continue that way one at a time.\sidenote{Sometimes people start the natural numbers at zero instead of one. It doesn't make much difference.} There is no largest natural number. They are infinite, which means that even these seemingly simple numbers are already quite complicated.

In ancient times, the Greeks restricted themselves to the numbers that could be constructed using a compass and an unmarked straight edge. They began with the natural numbers. Figure \ref{fig:mb-natural-numbers-construction} shows how you can produce any natural number, provided you have the time and paper.

First, starting a point, draw line segment with the straight edge. This line segment defines our unit length. The problem is how to produce another line segment of the same length as one you just drew. The straight edge, remember, is unmarked. So it doesn't help us out too much. The trick is to use the compass. For convenience, mark the starting point of line with the suggestive label \textsf{0} and the point at the other end with the label \textsf{1}.\sidenote{In Book \textsc{vii}, definition 1 of the \emph{Elements}, Euclid tells us that the first number is one, not zero. But let's follow E.W.~Djikstra's advice and start counting at zero anyway. \cite{Fitzpatrick2008, Djikstra1982}} We know that radii of a circle have the same length. So if we draw a circle centered at \textsf{1} that passes through \textsf{0}, it will contain all points that are all equally far away from \textsf{1} as the point \textsf{0} is. Breakthrough!

Now we can extend the original line segment to meet the circle. The point of their intersection is exactly one unit length away from the point \textsf{1}. Mark this point with the suggestive label \textsf{2}. We can continue this way as many times as we like---geometric line segments stretch as far as we care to imagine. After repeating the process $n$ times, we have constructed a line segment $n$ units long. Do it once again, and we're at $n+1$.

\begin{marginfigure}
  \begin{scaletikzpicturetowidth}{\textwidth}
\begin{tikzpicture}[scale=\tikzscale]
\foreach \x/\xcolor in {3.75/cyan, 6.25/black}
  \draw[shift={(\x,1)}, thick, color=\xcolor] (0,0) circle (1);

\draw[shift={(6.25,1)}, thick, color=cyan] (0,0) -- (0,1);
\draw[shift={(6.25,1)}, color=cyan, fill=cyan] (0,1) circle (1pt);

\foreach \x/\xcolor in {1.5/cyan, 3.75/black, 6.25/black} {
  \draw[shift={(\x,0)}, thick, color=\xcolor] (0,0) -- (0,1);
  \draw[shift={(\x,0)}, color=\xcolor, fill=\xcolor] (0,1) circle (1pt);
}

\foreach \x/\y in {0/0, 1.5/1, 6.25/2}
  \draw[shift={(\x,\y)}, fill=cyan, color=cyan] (0,0) circle (1pt) node[above] {\small\textsf{\y}};

\foreach \x in {1.5, 3.75, 6.25}
  \draw[shift={(\x,0)}, fill=black] (0,0) circle (1pt);

\draw[shift={(8.25,0)}, thick] (0,4) circle (1) node[above, left] {\small\textsf{n}};
\draw[shift={(8.25,0)}, thick, color=cyan] (0,4) -- (0,5);
\draw[shift={(8.25,0)}, color=cyan, fill=cyan] (0,5) circle (1pt) node[above] {\small\textsf{n+1}};
\draw[shift={(8.25,0)}, thick] (0,0) -- (0,4);
\foreach \x in {0,1,3,4}
  \draw[shift={(8.25,0)}, fill=black] (0,\x) circle (1pt);

\end{tikzpicture}
\end{scaletikzpicturetowidth}

  \caption{\label{fig:mb-natural-numbers-construction} Construction of the natural numbers}
\end{marginfigure}

There is something curious about the construction of the natural numbers given above: we got to pick the unit length in the first step. It didn't matter if we chose an inch, 19 centimeters, a dozen poronkusemas\sidenote{\emph{Poronkusema} is Finnish for the distance a reindeer can generally travel before stopping to urinate. It's not a precise measure of distance, perhaps 7.5 kilometers. But it's surely not more.}, or anything other length; the contstruction continues just same. Each choice gives a different \emph{coordinatization} of the line. \emph{Coordinates} are labels we give to points in geometric objects. In this case, the points on a line. Once we marked our zero point and chose a unit length, the positions of all of the coordinates of all the other points on the line were locked into place. Because our construction of the natural numbers did not make use of any particular coordinate system---remember, we got to choose the unit length---we say that the construction is \emph{coordinate-free} or \emph{coordinate independent}.

\begin{question}
  Figure \ref{fig:mb-3-real-lines} shows three real lines, each of which has been coordinatized by a different unit length. How are the coordinate systems related to one another?
\end{question}

\begin{marginfigure}
  \begin{scaletikzpicturetowidth}{\textwidth}
\begin{tikzpicture}[scale=\tikzscale]
\foreach \x in {0, 1, 2}
  \draw[shift={(0,\x)}, thick, <->] (0,0) -- (6,0);

\foreach \x in {-2,-1,2}
  \draw[shift={(3+\x,2)}, fill=black] (0,0) circle (1pt) node[below=3pt] {\small$\x$};
\draw[shift={(3,2)}, color=cyan, thick] (0,0) -- (1,0);
\draw[shift={(3,2)}, color=black, fill=black] (0,0) circle (1pt) node[below=3pt] {\small$0$};
\draw[shift={(3,2)}, color=black, fill=black] (1,0) circle (1pt) node[below=3pt] {\small$1$};

\foreach \x/\xlabel in {-2.5/-2, -1.25/-1, 2.5/2}
\draw[shift={(3+\x,1)}, fill=black] (0,0) circle (1pt) node[below=3pt] {\small$\xlabel$};
\draw[shift={(3,1)}, color=cyan, thick] (0,0) -- (1.25,0);
\draw[shift={(3,1)}, color=black, fill=black] (0,0) circle (1pt) node[below=3pt] {\small$0$};
\draw[shift={(3,1)}, color=black, fill=black] (1.25,0) circle (1pt) node[below=3pt] {\small$1$};

\foreach \x/\xlabel in {-2.5/2, 0.5/-1, 1.5/-2, 2.5/-3}
  \draw[shift={(3+\x,0)}, fill=black] (0,0) circle (1pt) node[below=3pt] {\small$\xlabel$};
\draw[shift={(3,0)}, color=cyan, thick] (-0.5,0) -- (-1.5,0);
\draw[shift={(3,0)}, color=black, fill=black] (-0.5,0) circle (1pt) node[below=3pt] {\small$0$};
\draw[shift={(3,0)}, color=black, fill=black] (-1.5,0) circle (1pt) node[below=3pt] {\small$1$};

\end{tikzpicture}
\end{scaletikzpicturetowidth}

  \caption{\label{fig:mb-3-real-lines} Three different coordinatizations by three different unit lengths}
\end{marginfigure}

\begin{question}
  Does Figure \ref{fig:mb-3-real-lines} prove that there are multiple, different real number lines? What do you think?
\end{question}

\begin{question}
  Consider a piece of string. Alice uses an English ruler to measure its length and finds the string to be exactly 25 inches long. Bob uses a metric ruler and finds the string to be exactly 63.5 centimeters long. Who is correct? What is the actual length of the string?
\end{question}

\begin{question}
  In what ways is the previous question about Alice and Bob like the geometric onstruction of the natural numbers?
\end{question}


You can do more than just construct the natural numbers. You can combine them! Two natural numbers $n$ and $m \in\mathbf{N}$ can be \emph{added} together to form their sum $n+m$ which is another natural number. To see the effects of addition geometrically, let's look at the specific of adding 1 to another number $n$. We can write this rule as

\begin{equation}
  \label{eqn:add-one}
  f: n \mapsto n + 1.
\end{equation}

We read equation \label{eqn:add-one} as the ``function $f$ that maps $n$ to $n + 1$.''\sidenote{Mathematical jargon is full of references to cartography. We have already bumped into maps and coordinates. Geometers deal with charts and altases, too. Maps capture relationships between the objects the represent, e.g., ``You are here. The nearest bathroom is there.'' Mathematics is the study of relationships between objects and transformations that preserve those relationships. So maps show up all the time.} We applied a function like $f$ when we first constructed the natural numbers, starting at the zero. When we apply $f$ to the whole real line, it looks like figure \ref{fig:add-one}.

\begin{question}
  Describe the effect of $f: n \mapsto n + 1$ on the real number line geometrically.
\end{question}

The two number lines in \ref{fig:add-one} are related to each other through the mapping $f$. The bottom lie is the \emph{image} of $f$. The image of a map is the set of output values $f(n)$ generated by applying the map to input values $n$. In this case, the image is a translation back one unit length by each coordinate. Informally, the whole number line got shifted back by one. Backward? That's a little confusing. Didn't we add to each point? Indeed we did.

\begin{marginfigure}
  \begin{scaletikzpicturetowidth}{\textwidth}
\begin{tikzpicture}[scale=\tikzscale]
\foreach \x in {0, 1.5}
  \draw[shift={(0,\x)}, thick, <->] (-3,0) -- (3,0);

  \foreach \x in {-2,...,2}
    \draw[shift={(\x,1.5)}, fill=black] (0,0) circle (1pt) node[below] {\small $\x$};
  \foreach \x in {-1,...,3}
    \draw[shift={(\x-1,0)}, fill=black] (0,0) circle (1pt) node[below] {\small $\x$};

% \node [draw, color=cyan] (A) at (4,5) {AAAA};
\node[color=cyan, above=3pt] (N) at (-2.5,1.5) {\small $n$};
\node[color=cyan] [below=1.2 of N.west, anchor=west] {\small $f(n) = n + 1$};
\end{tikzpicture}
\end{scaletikzpicturetowidth}

  \caption{\label{fig:add-one} The image of $f: n \mapsto n + 1$}
\end{marginfigure}

By adding 1 to each coordinate, we sped up the labeling of each point on the line. Thanks to $f$, coordinates appear one unit earlier than before, making them appear to have moved backwards. In this way, we can think of $f$ acting not on points on the line directly, but simply transforming their coordinate labels from one to another. The points on the line stay in place. Their labels move. To emphasis this viewpoint, we call $f$ a \emph{coordinate transformation} of the real line.

In general, coordinate transformations of the form $t_m: n \mapsto n + m$ translate the line $m$ units backwards. There is a special translation $t_0: n \mapsto n + 0$ called the \emph{identity} translation.\sidenote{The identity translation is so important, we'll give it its own notation $id_+$.} The identity translation is laziest translation you can imagine. It accepts an input $n$ and produces an output which is identically $n$. Because zero is the unique number such that $n + 0 = n$ for any $n \in \mathbf{N}$, we say that zero is the \emph{additive identity element} of the natural numbers.

\begin{question}
  Draw the image of $id_+$.
\end{question}

Notice that when $m \in \mathbf{N}$, the translation $t_m$ maps the coordinates of one natural number to the coordinate of another natural number. That is, the image $t_m(\mathbf{N})$ is the again the natural numbers $\mathbf{N}$. That is, the natural numbers are \emph{closed} under addition.\sidenote{I told you we'd see more examples of closure!}

\begin{question}
  What are some other operations that the natural numbers are closed under?
\end{question}

\begin{question}
  What is an operation that the natural numbers are not closed under?
\end{question}

% Closure under multiplication. (Scales things!) One as multiplicative identity. But $x + 1 = 0$ has no solution in the natural numbers. I.e., not closed under substraction. No, additive inverse. Enter the integers.

%
% \textbf{STREAM OF CONSCIOUSNESS}
%
% Then fractions. Pythagoras' cult discovered the irrationality of square root of two. Are there any holes? Answer to this question can be found in the study of real analysis. Basically all the numbers we would ever want are there. But special note: the reals are not the only completion of $\mathbf{Q}$. See, for example, the $p$-adics. For now, the real line will suffice.  We will review a few properties of the real numbers.
%
% Can we say \emph{the} real line with certainty? Field-preserving homomorphisms. Affine group. But real proof requires a proper construction, e.g., Dedekind cuts.
%
%
% \section{Complex Numbers}

% \marginnote{\textsf{Jordan Chapter 2: Imaginary Numbers}}
