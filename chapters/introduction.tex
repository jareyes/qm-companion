\setchapterpreamble[u]{\margintoc}
\chapter{Introduction}

This book is a companion to \emph{Quantum Mechanics in Simple Matrix Form} \cite{Jordan2005} by Thomas Jordan. Quantum mechanics was developed using to different, but equivalent mathematical formalisms: wave mechanics and matrix mechanics. Today many introductions to the subject follow the approach invented by Erwin Schr\"odinger and begin with waves, which requires knowledge of partial differential equiations. Using Jordan's book as a guide, we're going to follow in the footsteps of Werner Heisenberg and Max Born and develop quantum mechanics using matrix algebra.

Matrix algebra is a nice way to learn quantum mechanics for a few reasons. Firstly, it requires a lower entry to barrier on the mathematical front. The arithmetic of matrices builds on addition and multiplication of real numbers. Secondly, matrix mechanics generalizes in a particularly convenient way. Unlike real numbers, matrix products do not always commute \sidenote{Remember this fact. Commutation is one of the great big ideas in quantum mechanics.}. That is, given any two real numbers $x$ and $y$, we can multiply $x$ by $y$ or multiply $y$ by $x$. Either way, we get the same result.

$$ x\cdot y = y \cdot x $$

Given two matrices $A$ and $B$, the product $A$ applied to $B$ is not generally the same as the product $B$ applied to $A$.

$$ AB \neq BA $$

You'll notice that I wrote \emph{applied to} instead of \emph{multiplied by}, when I talked about the product of the matrices $A$ and $B$. Matrices are examples of \emph{linear operators}. Over time we will come to think of linear operators as representing observable quantites in laboratory experiments such as position, velocity, or momentum.\sidenote{You can squirrel this idea away for now, too. Linear operators represent observable quantities. I promise it will make sense later. It won't now. But we haven't even started yet.} But for now, you can think of them as a generalization of real numbers in which the order of operations matters. The operator formalism of quantum mechanics puts the commutation of observable quantities front and center. And to study quantum field theory, we'll need to graduate from state vectors to operators.

While Jordan wrote his book to be accessible to students with little mathematical background, his book moves pretty quickly. He requires what people often call ``mathematical maturity.''\sidenote{Said another way, there's very little that is simple in ``simple matrix form.''} This companion text aims to fill in some of the gaps that Jordan takes for granted. Along the way, we'll explore some of the mathematically structures associated with the physics that Jordan doesn't have the space to introduce himself. He used this book to teach a single-semester course, and it's already packed with more than enough to fill a full semester.

So without anymore ado, let's get to it!

\section{Mathematical Background}

The character of quantum mechanics is to a large extent dependent on the mathematics used to describe it. At the end of the day, all theory needs to be supported by experimental evidence. That is, by actual measurements. We quantify our measurements with numbers to speak in a standardized, precise way. When we ask questions like, ``How long?'' or ``How much?'', we typically use numbers to give an answer.

In quantum mechanics, we will use increasingly sophisicated mathematical constructs to describe the world around us, including experiments physicists perform. Some of properties of numbers that we take for granted will remain, while others will disappear. Here we begin with a exploration of two particular useful number systems: the \emph{real} and the \emph{complex} numbers.

\section{Real Numbers}

\begin{scaletikzpicturetowidth}{\textwidth}
\begin{tikzpicture}[scale=\tikzscale]
\draw[thick, <->] (-3.5,0) -- (3.5,0);
\foreach \x in  {-3,-2,-1,0,1,2,3,-0.333, 0.5, 1.4142, 2.7183, 3.1416}
\draw[shift={(\x,0)}, fill=black] (0,0) circle (1pt);
\foreach \x in {-3,-2,-1,0,1,2,3}
\draw[shift={(\x, 0)}] node[below=3pt] {\small $\x$};
\foreach \x/\xlabel in {-0.333/{-1/3}, 0.5/{1/2}, 1.4142/\sqrt{2}, 2.7183/e, 3.1416/\pi}
\draw[shift={(\x, 0)}] node[above=3pt] {\small $\xlabel$};
\end{tikzpicture}
\end{scaletikzpicturetowidth}


Real numbers. Communitivity, distributivity. (Figures showing geometric equivalence.)

Exercises.

Field properties. Example, the real numbers. Binary. Finite fields. $\mathbb{Z}_2$. $V_4$.

Exercises.

Cartesian plane. Absolute value. Distance. Circles.

Exercises.

\section{Complex Numbers}

\marginnote{\textsf{Jordan Chapter 2: Imaginary Numbers}}
