\setchapterpreamble[u]{\margintoc}
\chapter{Mathematical Background}

The character of quantum mechanics is to a large extent dependent on the mathematics used to describe it. At the end of the day, all theory needs to be supported by experimental evidence. That is, by actual measurements. We quantify our measurements with numbers to speak in a standardized, precise way. When we ask questions like, ``How long?'' or ``How much?'', we typically use numbers to give an answer.

In quantum mechanics, we will use increasingly sophisicated mathematical constructs to describe the world around us, including experiments physicists perform. Some of properties of numbers that we take for granted will remain, while others will disappear. Here we begin with a exploration of two particular useful number systems: the \emph{real} and the \emph{complex} numbers.

\section{Real Numbers}\marginnote{Real numbers. Communitivity, distributivity. (Figures showing geometric equivalence.) \textbf{Exercises.} Field properties. Example, the real numbers. Binary. Finite fields. $\mathbf{Z}_2$. $V_4$. \textbf{Exercises.} Cartesian plane. Absolute value. Distance. Circles. \textbf{Exercises.}}

We start of with our old friend, the \emph{real line}. The real line is an ordinary geometric straight line. We label the points on the line with the set $\mathbf{R}$ of all \emph{real numbers}. That is, we identify each point on the line with a unique real number. And  we often speak as if the numbers were actually on a line, and as if the points on the line are actually numbers.

\begin{figure}[h]
\begin{scaletikzpicturetowidth}{\textwidth}
\begin{tikzpicture}[scale=\tikzscale]
\draw[thick, <->] (-3.5,0) -- (3.5,0);
\foreach \x in  {-3,-2,-1,0,1,2,3,-0.333, 0.5, 1.4142, 2.7183, 3.1416}
\draw[shift={(\x,0)}, fill=black] (0,0) circle (1pt);
\foreach \x in {-3,-2,-1,0,1,2,3}
\draw[shift={(\x, 0)}] node[below=3pt] {\small $\x$};
\foreach \x/\xlabel in {-0.333/{-1/3}, 0.5/{1/2}, 1.4142/\sqrt{2}, 2.7183/e, 3.1416/\pi}
\draw[shift={(\x, 0)}] node[above=3pt] {\small $\xlabel$};
\end{tikzpicture}
\end{scaletikzpicturetowidth}

\caption{The real line}
\end{figure}

The structure of the real line is anything but simple. Do not be fooled. You may wonder what makes the real numbers ``real''? Begin with integers.

The Greeks restricted themselves to the numbers that could be constructed using a compass and an unmarked straight edge. The began with the so-called \emph{natural numbers}, usually denoted as a set by $\mathbf{N}$. The construction went like so [FIGURE]. Because the straight edge had no unit measure on it, the results they produced are \emph{coordinate-free}; that is, true no matter which coordinates we choose, be it inches, centimeters, or pikas.

Then fractions. Pythagoras' cult discovered the irrationality of square root of two. Are there any holes? Answer to this question can be found in the study of real analysis. For now, the real line will suffice.  We will review a few properties of the real numbers.

\begin{marginfigure}
\begin{scaletikzpicturetowidth}{\textwidth}
\begin{tikzpicture}[scale=1]
\draw[shift={(0,0)}, thick, color=cyan] (0,0) -- (0.5,0);
\draw[thick, color=cyan, fill=cyan] (0.5,0) circle (1pt);

\draw[shift={(0,-1.5)}, thick] (0,0) -- (0.5,0);
\draw[shift={(0,-1.5)}, thick, color=cyan] (0.5,0) circle (0.5);
\draw[shift={(0,-1.5)}, fill=black] (0.5,0) circle (1pt);

\draw[shift={(0,-3)}, thick] (0,0) -- (0.5,0);
\draw[shift={(0,-3)}, thick, color=black] (0.5,0) circle (0.5);
\draw[shift={(0,-3)}, thick, color=cyan] (0.5,0) -- (1,0);
\draw[shift={(0,-3)}, fill=black] (0.5,0) circle (1pt);
\draw[shift={(0,-3)}, thick, color=cyan, fill=cyan] (1,0) circle (1pt);

% \draw[shift={(0,-4.5)}, thick] (0,0) -- (1,0);
% \draw[shift={(0,-4.5)}, thick, color=cyan] (1,0) circle (0.5);
% \foreach \x/\xlabel in {0.5/1, 1.0/2}
% \draw[shift={(\x,-4.5)}, fill=black] (0,0) circle (1pt);
%
% \draw[shift={(0,-6)}, thick] (0,0) -- (1,0);
% \draw[shift={(0,-6)}, thick, color=black] (1,0) circle (0.5);
% \draw[shift={(0,-6)}, thick, color=cyan] (1,0) -- (1.5,0);
% \foreach \x/\xlabel in {0.5/1, 1.0/2}
% \draw[shift={(\x,-6)}, fill=black] (0,0) circle (1pt);
% \draw[shift={(0,-6)}, thick, color=cyan, fill=cyan] (1.5,0) circle (1pt);

\draw[shift={(0,-7.5)}, thick] (0,0) -- (5,0);
\draw[shift={(0,-7.5)}, thick, color=black] (5,0) circle (0.5);
\draw[shift={(0,-7.5)}, thick, color=cyan] (5,0)-- (5.5,0);
\foreach \x/\xlabel in {0.5/1, 1.0/2, 1.5/3, 2/4, 2.5/, 3/, 4.5/{n-1}, 5/n}
\draw[shift={(\x,-7.5)}, fill=black] (0,0) circle (1pt);
\draw[shift={(0,-7.5)}, color=cyan, fill=cyan] (5.5,0) circle (1pt);
\end{tikzpicture}
\end{scaletikzpicturetowidth}

\caption{Construction of the natural numbers}
\end{marginfigure}



\section{Complex Numbers}

\marginnote{\textsf{Jordan Chapter 2: Imaginary Numbers}}
